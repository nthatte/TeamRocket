\documentclass{article}
\usepackage{mathptmx}
\usepackage{amsmath}
\usepackage{graphicx}        % standard LaTeX graphics tool
\usepackage{url}
\usepackage{hyperref}
\graphicspath{{./figures/}}
\usepackage[top=1in, bottom=1in, left=1in, right=1in]{geometry}
%\pagenumbering{gobble}

\title{18-776/24/776: Rocket Landing Project \\ Group: Team Rocket \quad Score: 423.43}
\author{Nitish Thatte and Shihyun Laila Lo}

\begin{document}
\maketitle

\section{Details on our submission:}
Our submitted controller has two components: The first component is an attitude
controller that quickly flips the rocket from an initial angled state to a
roughly vertical position. Once the rocket attitude is regularized, we switch
to an LQR controller (computed using the dynamics linearized about the goal
position) to follow a minimum trajectory from the current state to the goal
state.

\section{Any cool graphics/Equations:}
\subsection{Attitude Control}
(Insert equations for attitude controller)

\subsection{Minimum Jerk Trajectories}
A minimum jerk trajectory can be encoded by a $5^{\textrm{th}}$ order polynomial of the form
\begin{align}
    x(t) = a_0 + a_1 t + a_2 t^2 + a_3 t^3 + a_4 t^4 + a_5 t^5
\end{align}
The corresponding velocity, acceleration, and jerk of this trajectory are:
\begin{align}
     \dot{x}(t) &= a_1 + 2 a_2 t + 3 a_3 t^2 + 4 a_4 t^3 + 5 a_5 t^4 \\
    \ddot{x}(t) &= 2 a_2 + 6 a_3 t + 12 a_4 t^2 + 20 a_5 t^3
\end{align}
If the length of trajectory is T, then we can find the coefficients to meet
specified initial and final positions, velocities, and accelerations by solving
the following system of equations:
\begin{align}
    \begin{bmatrix} 1 & 0 &   0 &    0 &     0 &     0 \\
                    1 & T & T^2 &  T^3 &   T^4 &   T^5 \\
                    0 & 1 &   0 &    0 &     0 &     0 \\
                    0 & 1 &  2T & 3T^2 &  4T^3 &  5T^4 \\
                    0 & 0 &   1 &    0 &     0 &     0 \\
                    0 & 0 &   2 &   6T & 12T^2 & 20T^3 \\
    \end{bmatrix} 
    \begin{bmatrix} a_0 \\ a_1 \\ a_2 \\ a_3 \\a_4 \\ a_5 \end{bmatrix} =
    \begin{bmatrix} x(0) \\ x(T) \\ \dot{x}(0) \\ \dot{x}(T) \\ \ddot{x}(0) \\ \ddot{x}(T) \end{bmatrix}
\end{align}
Sometimes the generated trajectories will go through the ground and approach
the landing pad from below.  We remedied this problem in two ways. The first
was to check the generated trajectory, and if it passed through the ground,
split the trajectory into two parts so that at half of the length of the
trajectory, the rocket has made it half way to the landing pad. The second
method is to leave the initial and final acceleration constraints unspecified, 
and instead find the minimum jerk trajectory by solving a quadratic program. In this method
the minimum jerk objective is encoded as a quadratic cost as follows:

\noindent The jerk of the z-trajectory is:
\begin{align}
     z^{(3)}(t) &= \begin{bmatrix} 0 & 0 & 0 & 6 & 24t & 60 t^2 \end{bmatrix} a_z \\
                &= w(t)^T a_z
\end{align}
Where $a_z$ is the vector of polynomial coefficients for the z-trajectory.
Our quadratic objective is the minimum sum of squared jerks:
\begin{align}
    \min_{a_z} & \ \sum \left(z^{(3)}(t)\right)^2\\
    \min_{a_z} & \ \sum (w(t)^Ta_z)^T(w(t)^Ta_z) \\
    \min_{a_z} & \  a_z^T \left(\sum w(t)w(t)^T \right) a_z 
\end{align}
We minimize this objective subject to the equality constraints given by the desired
initial and final positions and velocities and inequality constraints of the form:
\begin{align}
    z(t) \ge z(t)_{min}
\end{align}
Ultimately, we used the first method to generate our trajectory as it
surprisingly resulted in a higher score.

\section{Other methods we tried on the way:}
Our motivation for planning trajectories was that it would allow us to
precisely control the path the rocket takes. Therefore, we could prevent the
rocket from going through the ground or taking too long to reach the ground. We
also hoped to use these trajectoies with the iLQR procedure. In this procedure
the dynamics are linearized around the trajectory instead of the goal, and a
finite-time-horizon LQR controller is computed to follow the trajectory. The
system is then forward simulated resulting in a new trajectory. The dynamics
are then linearized about this new trajectory and the process is repeated. The end
result is a dynamicly feasable trajectory and a controller to follow it. As we
had access to the simulator code, we thought this was a promising approach to take.
Unfortunately, we couldnt achieve good results with this procuedure, either due
to a bug in the code or the nature of the rocket dynamics/saturations. In the end
we found regular LQR followed the trajectories well if the rocket attitude was 
regularized first.

\section{How our method changed after the Initial Conditions were known:}
\section{What we learned from this project:}
\section{References we found very useful for this project:}
\nocite{minjerk}
\nocite{acrl}
\bibliographystyle{IEEEtran}
\bibliography{citations}

\section{Feedback on improving the project next time:}
\end{document}
